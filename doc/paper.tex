\documentclass{article}

\usepackage[a4paper]{geometry}
\usepackage{microtype}
\usepackage{acronym}
\usepackage[standard]{ntheorem}
\usepackage{unicode-math}
\usepackage{haskell}
\usepackage{hyperref}
\usepackage{cleveref}

\author{Justin Bedő}
\title{Bioshake: a Haskell EDSL for bioinformatics pipelines}

\acrodef{edsl}[EDSL]{Embedded Domain Specific Language}

\begin{document}

\maketitle

\section{Introduction}

Bioinformatics pipelines are typically composed of a large number of programs
and stages coupled together loosely through the use of intermediate files.
These pipelines tend to be quite complex and require a significant amount of
computational time, hence a good pipeline must be able to
manage intermediate files, guarantee rentrability -- the ability to re-enter
a partially run pipeline and continue from the latest point -- and also provide
methods to easily describe pipelines.

\paragraph{Strong type-checking}

We present a novel bioinformatics pipeline framework that is composed as an
\ac{edsl} in Haskell. Our framework leverages the strong type-checking
capabilities of Haskell to prevent many errors that can arise in the
specification of a pipeline. Specifically, file formats are statically checked
by the type system to prevent specification of pipelines with incompatible
intermediate file formats. Furthermore, tags are implemented through Haskell
type-classes to allow metadata tagging, allowing various properties of files --
such as weather a bed file is sorted -- to be statically checked. Thus, a
miss-specified pipeline will simply fail to compile, catching these bugs well
before the lengthy execution.

\paragraph{Intrinsic and extrinsic building}

Our framework builds upon the shake \ac{edsl}, which is a make-like build tool.
Similarly to make, dependencies in shake are specified in an \textit{extrinsic}
manor, that is a build rule will define its input dependencies based on the
output file path. Our \ac{edsl} compiles down to shake rules, but allows the
specification of pipelines in an \textit{intrinsic} fashion, whereby the
processing chain is explicitly stated and hence no filename based dependency
graph needs to be specified. However, as bioshake compiles to shake, both
extrinsic and intrinsic rules can be mixed, allowing a choice to be make to
maximise pipeline specification clarity.

Furthermore, the use of explicit sequencing for defining pipelines allows
abstraction away from the filename level: intermediate files can be
automatically named and managed by bioshake, removing the burden of naming the
(many) intermediate files.

\begin{example}
  The following is an example of a pipeline expressed in the bioshake \ac{edsl}:
  \begin{haskell*}
    align &↦& sort &↦& dedup &↦& call &↦& out [''output.vcf''] 
  \end{haskell*}
  From this example it is clear what the \textit{phases} are, and the names of
  the files flowing between phases is implicit and managed by bioshake. The
  exception is the explicitly named output, which is the output of the whole
  pipeline.
\end{example}

\section{Core data types}

Bioshake is build around the following datatype:
\begin{haskell*}
  \hskwd{data} a ↦ b \hswhere{(↦) :: a → b → a ↦ b}\\
  \hskwd{infixl} 1 ↦ 
\end{haskell*}
This datatype represents the conjunction of two phases $a$ and $b$.
As we are compiling to shake rules, the $Buildable$ class represents a way to
build thing of type $a$ by producing shake actions:
\begin{haskell*}
  \hskwd{class} Buildable a \hswhere{build :: a → Action ()}
\end{haskell*}
Finally, as we are abstracting away from filenames, we use a typeclass to
represent types that can be mapped to filenames:
\begin{haskell*}
  \hskwd{class} Pathable a \hswhere{paths :: a → [FilePath]}
\end{haskell*}

\section{Defining phases}

A phase -- for example $align$ing and $sort$ing -- is a type in this
representation. Such a type is an instance of $Pathable$ as outputs from the
phase are files, and also $Buildable$ as the phase is associated with some shake
actions required to build the outputs. We give a simple example of declaring a
phase that sorts bam files.

\begin{example}
  \label{ex:sort}
  Consider the phase of sorting a bed file using samtools. We first define a
  datatype to represent the sorting phase and to carry all configuration options
  needed to perform the sort:
  \begin{haskell*}
    data Sort = Sort
  \end{haskell*}
  
  This datatype must be an instance of $Pathable$ to define the filenames output
  from the phase. Naming can take place according to a number of schemes, but
  here we will opt to use hashes to name output files. This ensure the filename
  is unique and relatively short.
  \begin{haskell*}
    \hskwd{instance} Pathable a ⇒ Pathable (a ↦ Sort) \hswhere{
      paths (a ↦ \_) = \hslet{inputs = paths a}{[hash inputs +\!\!\!+ ".sort.bed"]}
    }
  \end{haskell*}
  In the above, \<hash :: Binary a ⇒ a → String\> could be a cryptographic hash
  function such as sha1 with base32 encoding. Many choices are appropriate here.

  Finally, we describe how to sort files by making $Sort$ an instance of
  $Buildable$:
  \begin{haskell*}
    \hskwd{instance} (Pathable a, IsBam a, Pathable (a ↦ Sort)) ⇒ Buildable (a ↦ Sort) \hswhere{
      build p@(a ↦ \_) = \hslet{[input] = paths a\\[out] = paths p}{
        cmd "samtools sort" [input] ["-o", out]
      }
    } 
  \end{haskell*}
  Note here that $IsBam$ is a precondition for the instance: the sort phase is
  only applicable to BAM files. Likewise, the output of the sort is also a BAM
  file, so we must declare that too:
  \begin{haskell*}
    \hskwd{instance} IsBam (a ↦ Sort)
  \end{haskell*}
  The tag $IsBam$ itself can be declared as the empty typeclass \<class IsBam
  a\>. See \cref{sec:tags} for a discussion of tags and their utility.
\end{example}

\section{Compiling to shake rules}

The pipelines as specified by the core data types are compiled to shake rules,
with shake executing the build process. The distinction between $Buildable$ and
$Compilable$ types are that the former generate shake $Action$s and the latter
shake $Rules$. The $Compiler$ therefore extends the $Rules$ monad, augmenting it
with some additional state:
\begin{haskell*}
  \hskwd{type} Compiler &=& StateT (S.Set [FilePath]) Rules\\
\end{haskell*}
The state here captures rules we have already compiled. As the same phases may
be applied in several concurrent pipelines (i.e., the same preprocessing may be
applied but different subsequent variant callers) the set of rules already
compiled must be maintained. When compiling a rule, the state is checked to
ensure the rule is new, and skipped otherwise. The rule compiler evaluates the
state transformer, initialising the state to the empty set:
\begin{haskell*}
  compileRules &::& Compiler () → Rules ()\\
  compileRules p &=& evalStateT p mempty
\end{haskell*}

A compilable typeclass abstracts over types that can be compiled:
\begin{haskell*}
  \hskwd{class} Compilable a \hswhere{compile :: a → Compiler ()}
\end{haskell*}
Finally, \<a ↦ b\> is $Compilable$ if the input and output paths are defined,
the subsequent phase \<a\> is $Compilable$, and \<a ↦ b\> is $Buildable$.
Compilation in this case defines a rule to build the output paths with
established dependencies on the input paths using the $build$ function. These
rules are only compiled if they do not already exist: 
\begin{haskell*}
instance (Pathable a, Pathable (a ↦ b), Compilable a, Buildable (a ↦ b)) ⇒ Compilable (a ↦ b) \hswhere{
  compile pipe(a ↦ b) = do \hsbody{
    let outs = paths pipe\\
    set ← get\\
    when (outs `S.notMember` set) \$ do \hsbody{
      lift \$ outs \textrm{\&\%>} \_ → do \hsbody{
        need (paths a)\\
        build pipe}
      put (outs `S.insert` set)}
    compile a}}
\end{haskell*}

\section{Tags}
\label{sec:tags}

Bioshake uses a number of tags to ensure type errors will be raised if phases
are not connectable. There are numerous tags that are useful for capturing
metadata that are not otherwise easily captured. We have already seen in
\cref{ex:sort} the use of \textit{IsBam} to ensure the input file format of
\textit{Sort} is compatible. By convention, Bioshake uses the file extension
prefixed by \textit{Is} as tags for filetype, e.g.,: \textit{IsBam},
\textit{IsSam}, \textit{IsBcf}, \textit{IsBed}, \textit{IsCSV},
\textit{IsFastQ}, \textit{IsGff}, \textit{IsMPileup}, \textit{IsSam},
\textit{IsTSV}, \textit{IsVCF}.

Other types of metadata such as if a file is sorted (\textit{Sorted}) or if
duplicate reads have been removed (\textit{DeDuped}) are used. These tags allow
input requirements of sorting or deduplication to be captured when defining
phases. These properties can also automatically propagate down the pipeline; for
example, once a file is \textit{DeDuped} all subsequent outputs carry the
\textit{DeDuped} tag:
\begin{haskell*}
  \hskwd{instance} Deduped a ⇒ Deduped (a ↦ b)
\end{haskell*}

Finally, the tags discussed so far have been empty type classes, however tags
can easily carry more information. For example, bioshake uses a
\textit{Referenced} tag to represent the association of a reference genome. This
tag is defined as
\begin{haskell*}
  \hskwd{class} Referenced \hswhere{
    getRef :: FilePath
  }\\\\
  \hskwd{instance} Referenced a ⇒ Referenced (a ↦ b)
\end{haskell*}
This tag allows phases to extract the path to the reference genome and
automatically propagates down the pipeline.

\section{Abstracting execution platform}

In reality, it can be quite desirable to abstract the execution platform from
the definition of the build. In \cref{ex:sort}, the shake function $cmd$ is
directly used to execute samtools and perform the build. However, it is useful
to abstract away from $cmd$ directly to allow the command to be executed instead
on (say) a cluster, cloud service, or remote machine. Bioshake achieves this
flexibility by using free monad transformers to provide a function $run$ -- the
equivalent of $cmd$ -- but where the actual execution may take place via
submitting a script to a cluster queue, for example.

In order to achieve this, the datatype for phases in bioshake are augmented by a
free parameter to carry implementation specific default configuration -- e.g., cluster
job submission resources. In the running example of sorting a bed file, the new
datatype is \< \hskwd{data} Sort c = Sort c\>.

\section{Reducing boilerplate}

Much of the code necessary for defining a new phase can be automatically written
through the use of template Haskell. This allows very succinct definitions of
phases increasing clarity of code and reducing boilerplate. Bioshake has
template Haskell functions for generating instances of \textit{Pathable} and
\textit{Buildable}, and for managing the tags.

\begin{example}
  \Cref{ex:sort} can be simplified by using template Haskell considerably. First
  we have the augmented type definitions:
  \begin{haskell*}
    \hskwd{data} Sort c = Sort c
  \end{haskell*}
  The instances for \textit{Pathable} and the various tags can be generated with
  the template Haskell splice
  \begin{haskell*}
    \$(makeTypes ''\!Sort [''\!IsBam, ''\!Sorted] [])
  \end{haskell*}
  This splice generates a \textit{Pathable} instance using the hashed path
  names, and also declares the output to be instances of \textit{IsBam} and
  \textit{Sorted}. The first tag in the list of output tags is used to determine
  the file extension. The second empty list allows the definition of
  \textit{transient} tags; that is the tags that if present on the input paths will
  hold for the output files after the phase.
  Finally, given a generic definition of the build
  \begin{haskell*}
    buildSort t \_ (paths → [input]) [out] = \hsbody{run "samtools sort" [input] ["-@", show t] ["-o", out]}
  \end{haskell*}
  the \textit{Buildable} instances can be generated with the splice
  \begin{haskell*}
    \$(makeThreaded ''\!Sort [''\!IsBam] '\!buildSortBam)
  \end{haskell*}
  This splice takes the type, a list of required tags for the input, and the
  build function. Here, the build function is passed the number of threads to
  use, the \textit{Sort} object, the input object and a list of output paths.

\end{example}

\section{Discussion}

\end{document}

% Local Variables:
% TeX-engine: luatex
% End: